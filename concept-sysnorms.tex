\section{Norms of Signals and Systems}
\subsection{Norms}

Ingredients of a norm:
\begin{itemize}
	\item A linear space $V$ over $\mathbb C$ (or $\mathbb R$).
	\item A functional $$\norm{\cdot}\colon V \to \Rplus$$ that has the following properties:
		\begin{enumerate}
			\item $\norm{\alpha v} = \abs{\alpha}\norm{v}$,
			\item $\norm{v+w} \leq \norm{v}+\norm{w}$, and
			\item $\norm{v} = 0$ if, and only if, $v=0$,
		\end{enumerate}
		for any $v$, $w\in V$ and any $\alpha \in \mathbb C$ (or $\mathbb R$).
\end{itemize}

If $(V,\norm{\cdot}_V)$ and $(W,\norm{\cdot}_W)$, then for the space of linear maps $\bigl ( V \to W \bigr)$ a norm is defined via
$$
\norm{G}_* := \sup_{v\in V, v\neq 0}\frac{\norm{Gv}_W}{\norm{v}_V}.
$$
This is the norm for $G\colon V \to W$ that is induced by $\norm{\cdot}_V$ and $\norm{\cdot}_W$. There can be other norms that are not induced.

\subsection{Norms of Signals}
Common norms and spaces for the input and output signals 
$$u, y\colon [0, \infty) \to \mathbb R^{m}, \mathbb R^{q}$$:

The $\bL_1^m$ norm 
$$ \norm{u}_{\bL_1} := \int_0^\infty \sum_{i=1}^m\abs{u_i(t)} \inva t $$ 
that defines the $\bL_1^m$ space of integrable (summable) functions
$$\bL_1^m:=\bigl \{ u \colon [0, \infty) \to \mathbb R^{m} : \norm{u}_{\bL_1}<\infty \bigr \} $$

The $\bL_2^q$ norm 
$$ \norm{y}_{\bL_2} := \bigl(\int_0^\infty \sum_{i=1}^q\abs{y_i(t)}^2 \inva t\bigr)^{\frac 12} $$ 
that defines the $\bL_2^q$ space of square integrable functions
$$\bL_2^q:=\bigl \{ y \colon [0, \infty) \to \mathbb R^{q} : \norm{u}_{\bL_1}<\infty \bigr \} $$

The $\bL_\infty^m$ norm 
$$ \norm{u}_{\bL_\infty} := \max_{i=\{1, \dotsc, m\}} \sup_{t>0}\abs{u_i(t)} $$ 
that defines the $\bL_\infty^m$ space of bounded functions
$$\bL_\infty^m:=\bigl \{ u \colon [0, \infty) \to \mathbb R^{m} : \norm{u}_{\bL_\infty}<\infty \bigr \} $$

Where it is clear from the context, we will drop the superscripts $p$ and $m$ that denote the dimension of the signals.

The $\bL_2 $ norm can also be evaluated in frequency domain

\begin{theorem}
	For $u \in \bL_2 $ it holds that 
	\begin{equation*}
		\norm{u}_{\bL_2 } = \bigl ( \frac{1}{2\pi} \int_{-\infty}^{\infty} U(i\omega)^*U(i\omega) \inva \omega \bigr)^{\frac 12},
	\end{equation*}
	where $U$ is the \emph{Fourier} transform of $u$.
\end{theorem}

\begin{footnotesize}
	Fourier transform $\mathcal F$ and Laplace transform $\mathcal L$ coincide for $s=i\omega$, $\omega \in \mathbb R^{}$ and  $u(t)=0$ for $t\leq 0$:
	$$
	\mathcal F(u)(i\omega) := \int_{-\infty}^{\infty} u(t) e^{-i\omega t}\inva t = \int_0^{\infty} u(t)e^{-st}\inva t = \mathcal L(u)(s)
	$$
\end{footnotesize}
\subsection{Norm of a System}
A system $G$ or \abcdsys transfers inputs to outputs.
\begin{itemize}
	\item What does a norm mean for a system?
	\item Or, what is a large system, what is a small system?
\end{itemize}
From the definition of an operator norm.

$$
\norm{G} = \sup_{u\neq 0}\frac{\norm{Gu}}{\norm{u}} 
$$
meaning that for all $u$:
$$
\norm{y}=\norm{Gu}\leq \norm{G}\norm{u}.
$$

For systems, large refers to what extend an input is amplified and $\norm{G}$ is called the \emph{gain}.

Furthermore, with a norm one can compare two systems $G_1$ and $G_2$ via the difference in the output for the same input

$$\norm{y_1 - y_2 }=\norm{G_1u - G_2u}\leq \norm{G_1-G_2}\norm{u}.$$

\subsection{Defining a norm for systems}
Let $m=q=1$, i.e. we consider a SISO system \abcsys.

If \abcsys with stable and strictly proper transfer function $G$ is stable, then impulse response of the system
$$g(t) = C\int_0^t e^{A(t-s)}B\delta(s) \inva s = Ce^{At}B$$
decays exponentially and 
$$
\norm{g}_{\bL_2 } = \bigl( \frac{1}{2\pi} \int_{-\infty}^{\infty} G(i\omega)^*G(i\omega) \inva \omega \bigr)^{\frac 12} =: \norm{G}_2 < \infty
$$
This defines a norm for systems since (Exercise!)
\begin{itemize}
	\item $G=C(sI-A)^{-1}B$ is indeed the Laplace transform of $g$
	\item the functional $\norm{\cdot}_2$ for stable and strictly proper transferfunctions is a norm
\end{itemize}

Furthermore, (Exercise!), $\norm{y}_{\bL_\infty} \leq \norm{G}_2 \norm{u}_{\bL_\infty}$.

\begin{footnotesize}
	A system \abcdsys~or $A$ is stable, if there exists a $\lambda>0$, such that $\norm{e^At}\leq e^{-\lambda t}$, for $t>0$. This means that all eigenvalues of $A$ must have a negative real part.
\end{footnotesize}


For MIMO systems \abcsys~with $u(t)\in\Rm$ and $y(t)\in\Rq$, with a stable and strictly proper transferfunction $\mathcal G\colon s \to \mathbb R^{q\times m} $, the $\htwo$ norm is defined as  
\begin{equation*}
	\norm{G}_2 := \bigl( \frac{1}{2\pi} \int_{-\infty}^{\infty} \trace{ G(i\omega)^*G(i\omega)} \inva \omega \bigr)^{\frac 12}
\end{equation*}

This is the norm of the well-known $\emph{Hardy}$ space $\htwo$ of matrix functions that are analytic in the open right half of the complex plane. Stable and strictly proper transfer functions are in $\htwo$.

For a stable and proper transfer function one can define the $\hinf$ norm:
\begin{equation*}
	\norm{G}_\infty := \sup_{\omega \in \mathbb R^{} } \sigma_{\max{}} \bigl(G(i\omega)\bigr),
\end{equation*}
where $\sigma_{\max{}}\bigl(G(i\omega)\bigr)$ is the largest singular value of $G(i\omega)$.

This is the norm of the well-known $\emph{Hardy}$ space $\hinf$ of matrix functions that are analytic in the open right half of the complex plane and bounded on the imaginary axis. Stable and strictly proper transfer functions are in $\htwo$.

The $\hinf$-norm is induced by the $\bL_2$ norm:
\begin{equation*}
	\norm{G}_\infty = \sup_{u\in \bL_2, u\neq 0}\frac{\norm{Gu}_{\bL_2}}{\norm{u}_{\bL_2}} .
\end{equation*}
