\documentclass[a4paper,10pt]{article}
\usepackage[T1]{fontenc}
\usepackage[utf8]{inputenc}
\usepackage{epsfig}
\usepackage{amssymb,amsmath,amsthm,mathrsfs,mathtools,amscd}
\usepackage{natbib}
\usepackage{float}
\usepackage[figtopcap]{subfigure}
\usepackage{mymacros}
\usepackage{pgfplots}
% \pgfplotsset{compat=newest}
% \usetikzlibrary{external}
% \tikzexternalize[prefix=tikz/]
\usepackage[bookmarks=true]{hyperref}

% \input{mathComAbb} %uses mathcomabb.tex to define commands for symbols and notations 

%opening
\title{Introduction to LTI Systems}
\author{Jan Heiland}

\begin{document}
\maketitle
\tableofcontents
\def\abcsys{\ensuremath{(A,B,C,-)}}
\def\abcdsys{\ensuremath{(A,B,C,D)}}
% \def\abcsyst{\ensuremath{(TAT^{-1},TB,CT^{-1},-)}}
\def\abcdsyst{\ensuremath{(TAT^{-1},TB,CT^{-1},D)}}
\def\tfabcds{\ensuremath{C(sI-A)^{-1}B+D}}
\def\abcsyst{\ensuremath{(TAT^{-1},TB,CT^{-1},-)}}
\providecommand{\inva}[1]{\text{~\textup{d}} #1}
\providecommand{\norm}[1]{\lVert#1 \rVert}
\providecommand{\abs}[1]{\lvert#1\rvert}

\def\sco{_{co}}
\def\sbco{_{\bar c o}}
\def\sbcbo{_{\bar c \bar o}}
\def\scbo{_{c \bar o}}

\def\Rplus{\mathbb R_{\geq 0}}
\def\htwo{\ensuremath{\mathcal H_2}}
\def\hinf{\ensuremath{\mathcal H_\infty}}


\section{Examples}
\begin{itemize}
	\item Heat equation observation
\end{itemize}
\section{State Space Systems}
\begin{align*}
	Ex(t) &= Ax(t) + Bu(t), \quad x(0) = x_0, \\
	y(t) &= Cx(t),
\end{align*}
where 
\begin{itemize}
	\item $x(t) \in \Rn$: the system's state
	\item $u(t) \in \Rm$: the input or control
	\item $y(t) \in \Rq$: the output or measurements
	\item $E \in \Rnn$ is often the identity or the mass matrix of a discretization
	\item $A \in \Rnn$: the system matrix
	\item $B \in \Rnm$: the input matrix
	\item $C \in \Rqn$: the output matrix
\end{itemize}

We denote the LTI by $\abcsys$

Sometimes the state $x$ has a meaning in the model - like the temperature. Sometimes $x$ is not available or simply not of interest. Sometimes $x$ is just a mathematical object as a part of the model. 

\section{Transferfunctions}
The state is not accessible, not of interest or only an artificial object of the model for the input to output behavior $\mathbf G$ of a system $\mathbf P$
\begin{figure}[h]
\centering
\begin{tikzpicture}
 \draw [->] (3,1.1/2) -- (4,1.1/2); 
 \draw (4,0) rectangle (6,1.1) node[pos=.5] {$\mathbf P$};
 % \draw (4,0) rectangle (6,1.1) node[pos=.5] {\small{$y(t) = Cx(t), Ex(t) = Ax(t) + Bu(t)$}};
 \draw [->] (6,1.1/2) -- (7,1.1/2); 
 \node at (2.7,0.5) {$u$};
 \node at (7.3,0.5) {$y$};
\end{tikzpicture}
\end{figure}
that maps an input $u$ to the corresponding output $y$.

If $\mathbf P$ is an $\abcsys$, the function $\mathbf G$ can be defined via 
\begin{equation*}
	\mathbf P \colon u \mapsto y\colon y(t) = S(t, u)
\end{equation*}
this is time-domain. A function $u \colon [0, \infty) \to \Rm$ is mapped to a function $y\colon [0, \infty)  \to \Rq$.

Through the Laplace transform $\mathcal L$ and its inverse $\mathcal L^{-1}$, we can switch between time-domain and frequency-domain representations of the input and output signals.

\begin{equation*}
	U(s) := \mathcal L\{u\}(s) := \int_0^\infty e^{-st}u(t)\inva t,
\end{equation*}
where $s\in \mathbb C$ is the \emph{frequency} and
\begin{equation*}
	y(t) := \mathcal L^{-1}\{Y\}(t) := \lim_{T\to \infty} \frac 1{2\pi i} \int_{\gamma - iT}^{\gamma + iT} e^{s}Y(s)\inva s
\end{equation*}
where $\gamma \in \R$ is chosen such that contour path of the integration is the domain of convergence of $Y$.

With the basic properties of the Laplace transform
\begin{itemize}
	\item $\dot X(s):= \mathcal L\{\dot x\}(s) -x(0)= s\mathcal L\{x\}(s) = s X(s)-x(0)$
	\item and linearity $\mathcal L\{Ax\}(s) = AX(s)$
\end{itemize}
with zero initial value $x(0) = 0$, the $\abcsys$ defines the transferfunction
\begin{equation*}
	G(s) := C(sE-A)^{-1}B
\end{equation*}

\section{Realizations}
Given a rational matrix function $s\mapsto G(s)\in \mathbb R^{q\times m}$, is there $\abcsys$ so that 
\begin{equation*}
	G(s) = C(sI-A)^{-1}B ?
\end{equation*}
If there is one such state space system $\abcsys$, then there are infinitely many:

For every invertible $T$, the state space system $\abcsyst$ is a realization, since
\begin{equation*}
	C(sI-A)^{-1}B = CT^{-1}(s-TAT^{-1})^{-1}TB.
\end{equation*}

Also
\begin{equation*}
	(
	\begin{bmatrix} A & 0 \\ 0 & 0 \end{bmatrix},
	\begin{bmatrix} B \\ 0 \end{bmatrix},
	\begin{bmatrix} C & 0 \end{bmatrix}, -
	)
\end{equation*}
is a realization of one and the same transferfunction.

Thus, there are many possible state space representations of a transfer function. Which one is a good choice? The dimension can be arbitrary large. How small can it be? (cf. \emph{model reduction})

\section{Controllability, Observability}

\bibliographystyle{plain}
\bibliography{}
\end{document}

